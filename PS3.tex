\documentclass[12pt]{article}
\usepackage{fullpage}

\usepackage{graphicx}    	% Enable graphics commands
\usepackage{lscape}		% Enable landscape with \begin{landscape} until \end{landscape}
\usepackage{natbib}			% Enable citation commands \citep{}, \citet{}, etc.
\bibpunct{(}{)}{;}{a}{}{,}		% Formatting for in-text citations
\usepackage{setspace}		% Enable double-spacing with \begin{spacing}{2} until \end{spacing}.
\usepackage[utf8]{inputenc} 	% Enable utf8 characters, i.e., accents without coding--just type them in.
\usepackage[english]{babel}	% English hyphenation and alphabetization.  Other languages available.
\usepackage{dcolumn}        % For decimal-aligned stargazer output.
\usepackage[colorlinks=true, urlcolor=blue, citecolor=black, linkcolor=black]{hyperref} % Include hyperlinks with the \url and \href commands.
\setlength{\tabcolsep}{1pt}	% Make tables slightly narrower by reducing space between columns.

\renewcommand\floatpagefraction{.9}	% These commands allow larger tables and graphics to fit
\renewcommand\topfraction{.9}		% on a page when default settings would complain.
\renewcommand\bottomfraction{.9}
\renewcommand\textfraction{.1}
\setcounter{totalnumber}{50}
\setcounter{topnumber}{50}
\setcounter{bottomnumber}{50}

\newcommand{\R}{\textsf{R}~}        %This creates the command \R to typeset the name R correctly.

%\usepackage[left=1in, right=1in]{geometry}	%Turn footnotes into endnotes (commented out).
%\renewcommand{\footnotesize}{\normalsize}	
%\usepackage{endnotes}
%\renewcommand{\footnote}{\endnote}
%\renewcommand{\section}{\subsection}

\usepackage{Sweave}
\begin{document}
\Sconcordance{concordance:PS3.tex:PS3.Rnw:%
1 30 1 1 0 10 1 1 3 2 0 4 1 33 0 1 2 4 1 1 2 1 0 1 7 5 0 1 2 1 1 6 0 1 %
6 4 0 1 1 35 0 1 3 1 4 14 1 1 5 4 0 1 1 34 0 1 2 1 1 11 0 1 10 4 1 1 2 %
1 0 1 32 30 0 1 5 3 0 1 3 1 0 1 2 65 0 1 2 23 0 1 1 6 0 1 5 2 1 1 24 6 %
0 1 2 2 1 1 29 28 0 1 4 3 0 1 3 1 0 1 2 65 0 1 2 33 0 1 1 3 0 1 2 2 1 1 %
12 1 1 1 15 52 0 1 26 9 0 1 15 1 2 27 1}


\pagestyle{empty}

\begin{center}
{\Large \textbf{POLI 5003: Problem Set \# 1, Team A}}
\end{center}

The \emph{Partido Revolucionario Institucional} (PRI) maintained authoritarian rule over Mexico for more than seventy years, from the end of the Mexican Revolution until after the July 2000 elections.  The dataset accompanying this assignment (\texttt{mex2000.dta}) is drawn from a survey conducted during that electoral campaign.  You will use it to examine the predictors of Mexicans' attitudes towards the PRI and its opponents at that critical time in the country's history. 

\begin{Schunk}
\begin{Sinput}
> # Setup
> require(foreign)
> mex <- read.dta("mex2000.dta")
> var.labels <- attr(mex,"var.labels")
> data.key <- data.frame(var.name=names(mex),var.labels)
> data.key
\end{Sinput}
\begin{Soutput}
    var.name
1    PRIfeel
2    PANfeel
3    PRDfeel
4    prefPRI
5    prefPAN
6   rightide
7   econpers
8    econnat
9    corrupt
10     crime
11    female
12       ses
13 churchatt
                                                                         var.labels
1                          What is your opinion of the PRI? 0=very bad 10=very good
2                          What is your opinion of the PAN? 0=very bad 10=very good
3                          What is your opinion of the PRD? 0=very bad 10=very good
4                 PRIfeel - feeling toward best-liked opposition party (PAN or PRD)
5                                                                 PANfeel - PRIfeel
6                                    Political ideology, 0=very left, 10=very right
7  Change in personal economic situation, 1 yr. 1=much worse now, 5=much better now
8  View of national economic sit. over past yr. 1=much worse now, 5=much better now
9               View of gov't corruption, past yr. 1=much less now, 5=much more now
10                   View of crime over past year, 1=much less now, 5=much more now
11                                                              Female? 0=no, 1=yes
12                                    Socioeconomic status, 1=very low, 6=very high
13    Church attendance: 1=never, 2=occationally, 3=monthly, 4=weekly, 5=more often
\end{Soutput}
\end{Schunk}

\begin{enumerate}

\item During its long rule, the PRI worked to present itself as the party of all Mexicans and was therefore something of an ideological chameleon.  Nevertheless, we might hypothesize that people who leaned more to the right would hold more favorable views of this authoritarian party (Americanists may recall V.O. Key's writings about the one-party South), and suppose we want to control for their assessments of the recent performance of the national economy as well as of their personal characteristics.  Is this ideology hypothesis supported by a regression of \texttt{PRIfeel} using Empirical Bayes and the other default settings of MCMCpack?  How do you know?  Describe the estimated effect of ideology on attitudes toward the PRI.  \\

\begin{Schunk}
\begin{Sinput}
> require(MCMCpack)
> ipak <- function(pkg){
+      new.pkg <- pkg[!(pkg %in% installed.packages()[, "Package"])]
+      if (length(new.pkg)) 
+          install.packages(new.pkg, dependencies = TRUE)
+      sapply(pkg, require, character.only = TRUE)
+  }
> packages <- c("ggplot2", "RCurl", "MCMCpack", "inline", "Rcpp")
> ipak(packages)
\end{Sinput}
\begin{Soutput}
 ggplot2    RCurl MCMCpack   inline     Rcpp 
    TRUE     TRUE     TRUE     TRUE     TRUE 
\end{Soutput}
\begin{Sinput}
> m1<-MCMCregress(PRIfeel~rightide+econnat+female+ses+churchatt,data=mex,
+     burnin = 1000, mcmc = 10000,
+     thin = 1, verbose = 0, seed = 19880201, beta.start = NA, 
+     b0 = 0, B0 = 0, c0 = 0.001, d0 = 0.001, sigma.mu = NA, sigma.var = NA, 
+     marginal.likelihood =("none") )
> summary(m1)
\end{Sinput}
\begin{Soutput}
Iterations = 1001:11000
Thinning interval = 1 
Number of chains = 1 
Sample size per chain = 10000 

1. Empirical mean and standard deviation for each variable,
   plus standard error of the mean:

                Mean      SD  Naive SE Time-series SE
(Intercept)  1.67865 0.40282 0.0040282      0.0040282
rightide     0.28961 0.02682 0.0002682      0.0002678
econnat      0.79225 0.09087 0.0009087      0.0009087
female       0.53840 0.16174 0.0016174      0.0016174
ses         -0.23964 0.07272 0.0007272      0.0007095
churchatt    0.03758 0.07191 0.0007191      0.0007191
sigma2      10.38477 0.37170 0.0037170      0.0036131

2. Quantiles for each variable:

               2.5%      25%      50%      75%    97.5%
(Intercept)  0.8814  1.40950  1.67825  1.94583  2.46143
rightide     0.2370  0.27159  0.28963  0.30782  0.34258
econnat      0.6148  0.73190  0.79161  0.85248  0.97128
female       0.2197  0.43032  0.53729  0.64594  0.86219
ses         -0.3817 -0.28873 -0.24061 -0.19132 -0.09585
churchatt   -0.1027 -0.01148  0.03721  0.08571  0.17929
sigma2       9.6841 10.12877 10.37587 10.63067 11.12679
\end{Soutput}
\begin{Sinput}
> 
\end{Sinput}
\end{Schunk}

